% !TeX program = xelatex
\documentclass[12pt,a4paper]{ctexart}
\usepackage{amsfonts,amsmath,amssymb,graphicx,url}
\usepackage{fullpage}
\usepackage{enumitem}
\usepackage{multido}
\allowdisplaybreaks
\setlist[enumerate]{font=\bfseries}

% Old Stuff
%%\oddsidemargin=0.15in
%%\evensidemargin=0.15in
%%\topmargin=-.5in
%%\textheight=9in
%%\textwidth=6.25in

\setlength{\oddsidemargin}{.1in}
\setlength{\evensidemargin}{.1in}
\setlength{\topmargin}{-0.4in}

% From TeX-SE
\usepackage{enumitem}
\newcommand{\bitem}{\item\hspace*{1em}\ignorespaces}
\newenvironment{turing}[2]
{\begin{enumerate}[leftmargin=0pt,labelsep=0pt,align=left,parsep=0pt]
		\item[$#1={}$]``\ignorespaces#2
		\begin{enumerate}[nosep,align=left,labelwidth=1.5em,label=\bfseries\arabic{*}.,ref=\arabic{*}]}
		{\unskip''\end{enumerate}\end{enumerate}}

\newcommand{\heading}[5]{
   \renewcommand{\thepage}{\arabic{page}}
   \noindent
   \begin{center}
   \framebox{
      \vbox{
    \hbox to 6.2in { {\bf B62005Y-02 理论计算机科学基础}
     	 \hfill #2 }
       \vspace{4mm}
       \hbox to 6.2in { {\Large \hfill #5  \hfill} }
       \vspace{2mm}
       \hbox to 6.2in { {\it #3 \hfill #4} }
      }
   }
   \end{center}
   \vspace*{4mm}
}

\newcommand{\handout}[3]{\heading{#1}{#2}{主讲教师:杨光}{张远航 \textsc{2015K8009929045}}{#3}}

\setlength{\parindent}{0in}
\setlength{\parskip}{0.1in}

\def\eps {\varepsilon}
\def\x{\texttt{x}}
\def\a{\texttt{a}}
\def\b{\texttt{b}}
\def\ling{\texttt{0}}
\def\yi{\texttt{1}}
\def\jing{\texttt{\#}}
\def\kong{\sqcup}
\def\jujue{q_\mathrm{reject}}

\newcommand{\rep}{\multido{\i=1+1}}

\begin{document}
\handout{2}{\today}{Solutions to Homework \# 3}

\begin{enumerate}
\item[Sipser 3.1](d) $q_1\rep{6}{\ling}$, $\kong q_2\rep{5}{\ling}$, $\kong\x q_3\rep{4}{\ling}\kong$, $\kong\x\ling q_4\rep{3}{\ling}\kong$, $\kong\x\ling\x q_3\rep{2}{\ling}$, $\kong\x\ling\x\ling q_4\ling$, $\kong\x\rep{2}{\ling\x}q_3\kong$, $\kong\x\ling\x\ling q_5\x\kong$, $\kong\x\ling\x q_5\ling\x\kong$, $\kong\x\ling q_5\x\ling\x\kong$, $\kong\x q_5\rep{2}{\ling\x}\kong$, $\kong q_5\x\rep{2}{\ling\x}\kong$, $q_5\kong\x\rep{2}{\ling\x}\kong$, $\kong q_2\x\rep{2}{\ling\x}\kong$, $\kong\x q_2\rep{2}{\ling\x}\kong$, $\kong\x\x q_3\x\ling\x\kong$, $\kong\rep{3}{\x}q_3\ling\x\kong$, $\kong\rep{3}{\x}\ling q_4\x\kong$, $\kong\rep{3}{\x}\ling\x q_4\kong$, $\kong\rep{3}{\x}\ling\x\kong\jujue$.

\item[Sipser 3.2](d) $q_1\yi\ling\jing\yi\yi$, $\x q_3\ling\jing\yi\yi$, $\x\ling q_3\jing\yi\yi$, $\x\ling\jing q_5\yi\yi$, $\x\ling q_6\jing\x\yi$, $\x q_7\ling\jing\x\yi$, $q_7\x\ling\jing\x\yi$, $\x q_1\ling\jing\x\yi$, $\x\x q_2\jing\x\yi$, $\x\x\jing q_4\x\yi$, $\x\x\jing\x q_4\yi$, $\x\x\jing\x\yi\jujue$.

\item[Sipser 3.8](b) The following machine $M$ works:
\begin{turing}{M}{On input string $w$:}
	\item Scan the tape and mark the first \yi{} which has not been marked. If no unmarked \yi{} is found, go to step 5. Otherwise, move the head back to the front of the tape.
	\item Scan the tape and mark the first \ling{} which has not been marked. If no unmarked \ling{} is found, reject.
	\item Move on to mark the next unmarked \ling. If no such unmarked \ling{} is found, reject.
	\item Move the head back to the front of the tape and go to step 1.
	\item Move the head back to the front of the tape. Scan the tape to see if there is any unmarked \ling{} found. If yes, reject. Otherwise, accept.
\end{turing}

\item[Sipser 4.1]
\begin{enumerate}
	\item[(a)] Yes, because $M$ on input \ling\yi\ling\ling{} ends in an accept state.
	\item[(b)] No, because $M$ on input \ling\yi\yi{} ends in a non-accept state.
	\item[(c)] No, because the input is not in correct form: the second component of the input is missing.
	\item[(d)] No, because the input is not in correct form: the first component should be a regular expression but not a \textsf{DFA}.
	\item[(e)] No, because $M$ accepts $\eps$ and hence, $L(M)\ne\emptyset$.
	\item[(f)] Yes, because $L(M) = L(M)$.
\end{enumerate}
\item[Sipser 4.2]
The problem of testing whether a \textsf{DFA} and a regular expression are equivalent can be expressed by the following language:
\[EQ_{\mathsf{DFA}-\mathsf{REX}}=\{\langle M,r \rangle \mid M\text{ is a \textsf{DFA} and }r\text{ is a regular expression}\}\]
and $L(M) = L(r)$. We can prove that the language $EQ_{\mathsf{DFA}-\mathsf{REX}}$ is decidable by constructing a \textsf{TM} $P$ that decides it as follows:
\begin{turing}{P}{On input $\langle M,r \rangle$:}
	\item Convert the regular expression $r$ into a \textsf{DFA} $M_r$ by using the procedure described in Theorem 1.28.
	\item Apply the algorithm given in Theorem 4.5 to decide whether $\langle M,M_r \rangle\in EQ_{\mathsf{DFA}}$.
	\item If $\langle M,M_r \rangle\in EQ_{\mathsf{DFA}}$ then accept, else reject.
\end{turing}

\item[Sipser 4.3]
Let $M_{\Sigma^*}$ be a \textsf{DFA} that accepts $\Sigma^*$; then for every \textsf{DFA} $A$,
\[A\in ALL_{\mathsf{DFA}}\Leftrightarrow\langle A,M_{\Sigma^*} \rangle\in EQ_{\mathsf{DFA}}.\]
Therefore, to decide whether $A \in ALL_{\mathsf{DFA}}$, we just need to decide whether $\langle A,M_{\Sigma^*} \rangle\in EQ_{\mathsf{DFA}}$. The latter can be done by applying the proof in Theorem 4.5. Thus $ALL_{\mathsf{DFA}}$ is decidable.
\item[Sipser 4.4]
Since $A_{\eps\mathsf{CFG}}$ is just a special case of $A_{\mathsf{CFG}}$, it is possible to adapt $\mathsf{TM}$ $S$ for $A_{\eps\mathsf{CFG}}$ as follows.
\begin{turing}{S}{On input $(G,\eps)$, where $G$ is a \textsf{CFG} and $\eps$ is an empty string:}
	\item Convert $G$ to an equivalent grammar in Chomsky normal form.
	\item If ``$S\to\eps$" is a production rule in Chomsky normal form, accept; if not, reject.
\end{turing}
\end{enumerate}
\end{document}
